\chapter{Introdução}
\label{cap:01}

Aqui deve-se introduzir o trabalho. Trata-se da parte inicial do texto, onde devem constar: a delimitação do assunto tratado, os objetivos da pesquisa e outros elementos necessários para situar o tema do trabalho.
 
\section{Justificativa}

Texto da justificativa. Conte como se chegou ao "questionamento", como se identificou o problema, por que e de onde ele surgiu. É a explicação do porquê a inquietação em questão existe.

\section{Objetivos}
Descrever o que se pretende alcançar ao final do projeto.

\subsection{Objetivo Geral}

Qual seu objetivo geral.

\subsection{Objetivos Específicos}
\begin{itemize}
	\item Objetivo específico 1;
	\item Objetivo específico 2;
	\item Objetivo específico n.
\end{itemize}


\section{Problema (Hipótese ou Questão de Pesquisa)}

Escolha uma dentre as 3 formas de escrever: 1) o problema que o trabalho propõe resolver; ou 2) as hipóteses que se deseja comprovar (ou não) com o estudo; ou 3) as questões de pesquisa levantadas para serem investigadas.


\section{Organização Deste Trabalho}

Como seu trabalho está organizado (capítulos) após a introdução.